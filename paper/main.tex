% \documentclass[12pt,reqno]{amsart}
\documentclass{article}
% \usepackage[letterpaper, margin=1.1in]{geometry}
\usepackage{amsfonts} 
\usepackage{amsmath}
\usepackage{amsthm}
\usepackage{amscd}
\usepackage{amsfonts}
\usepackage{amssymb}
\usepackage{mathrsfs}
\usepackage[bookmarks, colorlinks=true, allcolors=blue]{hyperref}
\usepackage{graphicx, wrapfig}
\usepackage[usenames]{color}
\usepackage[top=1.25in, bottom=1.25in, left=1.25in, right=1.25in]{geometry}
% \usepackage{natbib}
%\usepackage{showkeys}
\usepackage{caption}
\usepackage{subcaption}
\usepackage{longtable}
\usepackage{dsfont}
\usepackage{wrapfig}
\usepackage{authblk}
\usepackage{booktabs}

\usepackage[sorting=none, style=numeric-comp]{biblatex}

\addbibresource{/Users/mt/workspace/Writing/library.bib}
\addbibresource{this.bib}


 \definecolor{red}{rgb}{1.0,0.0,0.0}
 \def\red#1{{\textcolor{red}{#1}}}
 \definecolor{blu}{rgb}{0.0,0.0,1.0}
 \def\blu#1{{\textcolor{blu}{#1}}}
 \definecolor{gre}{rgb}{0.03,0.50,0.03}
 \def\gre#1{{\textcolor{gre}{#1}}}
 \definecolor{darkviolet}{rgb}{0.58, 0.0, 0.83}
\def\dvio#1{{\textcolor{darkviolet}{#1}}}



% \parskip=5pt


\definecolor{mt-orange}{RGB}{240, 96, 0}
\newcommand{\mt}[1]{{\textcolor{mt-orange} {({\tiny MT:} #1)}}}
\definecolor{your-color-name}{RGB}{60,50,168}
\newcommand{\YourInitialsHere}[1]{{\textcolor{your_color_name} {({\tiny YI:} #1)}}}

\begin{document}

\title{Overcoming uncertainty and complexity, impediments to biodiversity protection in
the Greater Mara, Kenya, and similar ecosystems}

\author{Matthew A. Turner$^{*,1}$, SFI Biodiversity Team$^{2}$ \\ 
  {\small $^1$ Environmental Social Sciences, Stanford Doerr School of Sustainability, Stanford University, USA \\ \vspace{0.5em}
$^2$Earth} \\ \vspace{0.5em}
{\small $^*$Correspondence: \href{mailto:maturner@stanford.edu}{maturner@stanford.edu}}}
\maketitle

\begin{abstract}
  \noindent
  Over the past few decades, conservationists and policymakers have struggled
  to achieve biodiversity protection goals.  We suggest that this is because
  any decades-long policy for biodiversity protection is subject to
  uncertainty arising from complexty inherent in socio-ecological systems
  needing biodiveristy protection. For example, in the past several decades in
  the Greater Mara ecosystem, unregulated fence construction has reduced
  biodiversity protection  by blocking migratory corridors for animals like
  ungulates threatened with massive die-off if they cannot migrate to find
  water and food in drought conditions.  To deal with uncertainty and
  complexity, we propose a multi-level modeling approach that combines
  long-term, large-scale policy optimization with shorter-term, local-scale
  behavioral and ecological simulations.  At time scales of about a century,
  and ecosystem-level spatial scales, we propose using dynamic optimization
  for long-term planning. To complement this, we propose using agent-based
  models predict the effects of these policies on landholder decision making
  in explicitly-modeled socio-ecological contexts over time scales of about
  ten years and ~100 square-kilometer spatial scales. Using the Greater Mara
  as a case study, in this paper we specifically develop the agent-based 
  model component to demonstrate how scientific, mechanistic models can help
  design and evaluate interventions to reduce fencing and support biodiversity
  protection.
\end{abstract}


\section{Introduction: predicting effects of environmental policy on 
  biodiversity (in the Greater Mara, Kenya)}

Biodiversity has both intrinsic and economic value, but is increasingly threatened
due to expanding human populations, overexploitation of land and game animals,
climate change and pollution, and invasive species encroachment~\cite{Bellard2022}.
Government policy, social influence, and personal preference all affect land use
decisions on different time, population, and geospatial scales 
(plant crops, raise cattle, etc.)~\cite{Eitzel2020,Løvschal2021,Richerson2024}.

Fence construction is one important anthropocene feature that threatens
biodiversity~\cite{Packer2013} and whole 
ecosystems~\cite{Løvschal2017,Løvschal2022}. 
In turn, changes in biodiversity affect land use, resulting in a complex, coupled
dynamical system. In the Greater Mara ecosystem, in Keyna, 
ungulates are particularly threatened~\cite{Ogutu2009}, with fencing being a 
primary driver of declining biodiversity by cutting off otherwise accessible 
migration routes used in times of drought~\cite{Ogutu2016}. 
There are many reasons why people subdivide land and build
fences~\cite{Mwangi2007}, though these may change depending on local context.
Similarly, the risks posed by different fences are ostensibly
different in different contexts~\cite{Bellard2022}, perhaps even within
a single ecosystem like the Mara. 

Because of the need to understand land use and
fence construction on different time, population, and geospatial scales, we
outline here one component of a modeling strategy aimed at taming the 
complexity inherent in
biodiversity protection over long time scales. This modeling approach 
is intended to complement a long-term optimization 
approach that predicts outcomes over
time scales on the order of one century, and over large population and geospatial
scales on the order of an entire ecosystem. Such large-/long-scale approaches
could be used to predict and plan how government policies will affect the distribution of
land use, the size of wildlife populations, and the total area fenced over
several decades.
Of course these coarse-grained dynamics occur over decades and emerge from
individual-level decisions and local interpersonal interactions, such as 
when a farmer in the Mara must decide whether to build a fence or not based on
the amount of wildlife concentration on their land, which is a function of whether or
not neighbors built fences, joined conservancies, etc. 

Therefore, here, we outline a complementary, more localized agent-based
modeling framework to predict fence building, specifically, and its effect on
trophic dynamics and biodiversity. This is a proof-of-concept agent-based
model where individual-level incentives and interpersonal influence can be
incorporated to drive fence building decisions, which in turn affect
biodiversity, which in turn affects trophic network dynamics. 
Such an approach is necessary because we clearly cannot perform real-world
experiments on the Mara, or any other biodiversity preserve, because we do not
get more than one, and it must be saved now. Such mechanistic modeling enables
us to predict how the past or future may have been or could be different
depending on contextual legal, environmental, and behavioral
factors~\cite{Turner2022}. This approach provides a strategy for anticipating
a range of possible outcomes for biodiversity interventions by simulating the
complex local-, individual-, and relatively shorter time-scale dynamics of
stakeholder decision-making and trophic networks that are affected by
regional, state, or national policy. 

For instance, we may use this modeling approach to
demonstrate that a push to incorporate more rangeland into conservancies can tip
unincorporated pastoralists to build fences around their landholdings, which may
eventually limit wild herbivore mobility by encircling herbivores in the reserve,
which can become deadly. In one tragic example, in the case of a sudden severe
drought, many landholders had been fencing their land over the course of many
years, which had not been a problem since water had been plentiful. When the
drought hit, wildebeest needed to migrate for water, but were unable to due to
fence construction that only became an obvious problem after that tragic event.
Our modeling framework is intended to predict such tragedies before they occur. 
These more localized, shorter time-scale predictions could in turn be used to
update optimal policies, which in turn generate new smaller-scale simulation
outcomes, and so on, so that the models at each scale contribute to a strategy
for implementing long-term \emph{and} adaptive policy to protect biodiversity.

In the rest of this brief report I introduce this agent-based model. The
model represents ``decision making'' among dozens of landholders, and is
capable of representing hundreds or thousands of livestock and other non-human
animals existing over grids with tens or hundreds of thousands of cells that
can run over dozens of months in minutes or less on my MacBook Air laptop.
Model dynamics can be analyzed with custom-made movies of fence building and
prevalence dynamics in time series analyses. Despite these achievements, it
is still just a toy prototype model in the sense that fencing decisions are
made randomly and trophic and environmental dynamics have been made \emph{ad
hoc} in order to get some minimally interesting dynamics. With the guidance of
the Santa Fe group, and potentially other experts, this model can be made more
scientific, with modeling decisions based on more refined, agreed-upon
scientific questions.

\section{Model}
\label{sec:abm}

To understand how fencing decisions may increase the risk of biodiversity
loss, we developed an agent-based model representing vegetation, a simple
trophic network (Figure~\ref{fig:composition}A), landholders
(Table~\ref{tab:landholders}), their livestock, and, most importantly, their
fences. Over time, trophic dynamics play out, most notably with livestock and
other herbivores consuming grass and carnivores hunting and consuming
livestock and herbivores. Outcomes are analyzed by inspecting the prevalence
of different ``species'' over time (Figure~\ref{fig:composition}B) and 
monthly maps of vegetation availability and (Figure~\ref{fig:composition}C). 
These model features are explained in more detail in the rest of this section.


\subsection{Model components}

The model is composed, first, of landholders, who occupy plots of land, with
vegetation growing on each cell of a square grid. On this same grid, there are
five ``species'' of animals: ``carnivores'', ``livestock'', ``herbivores''
(free-roaming non-livestock), ``birds'', and ``insects'' (not shown in
analyses). Currently the birds and insects play no interesting role, but they
are included since the big picture goal is protecting biodiversity, and I
wanted to include more biodiversity in the trophic network.

\subsubsection{Landholders}

Landholders are agents who own a plot of land and livestock, and may have
enclose their land with a fence (pink polygons in
Figure~\ref{fig:composition}B). They are assigned a probability of building a 
fence on a given time step, which for this toy model is set constant for the
entire simulation. In the future this probability would be tied to
non-livestock herbiveore encroachment, carnivore predation of livestock, new
legislation, and social learning of other landholder's fencing decisions, e.g., 
through observation or direct instruction. 
Landholder variables are summarized below in Table~\ref{tab:landholders}.

\begin{table}[h]
  \caption{\textbf{Landholder variables.} Landholders are allotted a plot of
  land with livestock and can choose to fence or not.}
  \label{tab:landholders}
  \begin{tabular}{cl} \toprule
    variable & description \\ \midrule  
    \textsc{holding} & Plot of land where livestock live, held by landholder \\
    \textsc{livestock} & Collection of herbivores in holding owned by landholder \\
    \textsc{fenced} & Whether or not landholder's land is fenced \\
    \textsc{fence\_probability} & Baseline probability a landholder will build a
    fence around holding \\
    \bottomrule
  \end{tabular} 
\end{table}

\subsubsection{Vegetation and trophic network}

Trophic network composed of five ``species'': insect, bird, herbivore,
carnivore, and livestock. Of course livestock are a type of herbivore, but
we need to separate them so that if they are being hunted by carnivores,
for example, this can factor in to landowner decision making. 

All members of the trophic network have a location, ``energy'' 
they have to move or give to those who
consume them. In the code, each animal is an instance of the \texttt{Critter} 
class (see attributes of this class in Table~\ref{tab:nonhuman}). 
Livestock belong to a landholder, who own them
and the land on which they roam. Whether or not the landholder has fenced their
holding, livestock are constrained to stay on the landholder's land. If a fence
has been built, carnivores cannot predate the livestock, and non-livestock
herbivores cannot encroach on that land. Carnivores are initialized with a 
\emph{catch radius} that specifies how close prey must be for them to
successfully capture and consume it. Herbivores and livestock gain energy from 
eating grass. Herbivores and carnivores must eat sufficient amounts of
``energy'' in order to keep on living, otherwise they die. All animals
reproduce with a certain probability.

\begin{table}[h]
  \caption{\textbf{Non-human animal variables.} All non-human species have
  the share a set of attributes. Predators have an additional attribute, called
\textsc{catch\_radius}, specifying how close prey must be in order to catch it.}

  \label{tab:nonhuman}
  \begin{tabular}{cl} \toprule
    variable & description \\ \midrule  
    \textsc{position} & Spatial location of animal \\
    $\Delta$\textsc{energy} & Energy used to live for one time step \\
    \textsc{energy} & Amount of energy animal has left; animal dies when $<0$ \\
    \textsc{species} & Animal species: insect, bird, herbivore, carnivore, or
    livestock \\
    \textsc{landholder} & Pointer to landholder if animal is livestock; \texttt{NULL}
    otherwise \\
    \textsc{catch\_radius} & Distance away for which a prey animal can be caught; \texttt{NULL} for non-predators  \\
    \bottomrule
  \end{tabular} 
\end{table}

\subsection{Model dynamics}

The model is initialized with a grid of
``vegetation'', or ``grass''; a given number of animals of each model
species, a certain number of landholders, and each landholder either having
a fence or not, depending on the probability \texttt{initial\_fence\_prob}.
There are many model parameters necessary to initialize a model; for the
purposes of this brief report please see the \href{model initialization code
on GitHub}{https://github.com/mt-digital/fences} for a full listing.  

On each time step, meant to represent about one month, 
each landholder may choose to build a fence, which they do with
probability \texttt{fence\_probability}. No animals may pass through a
fence except for birds and insects. 
Every animal attempts to move to a new place on the grid.
Herbivores and livestock eat whatever grass is available. Grass grows back as
time progresses. Carnivores eat
one herbivore or livestock if there are any within the
\texttt{catch\_radius}. If there are more than one within the catch
radius, one is chosen at random to be consumed. 

To represent a drought, some percentage (currently 75\%) of all grass cells go
0 grass (as occurs at $t=42$ in Figure~\ref{fig:composition}C).


\subsection{Outcome measures}

Currently the primary outcome measures are the prevalence of each animal in the
grid, which is implicitly a measure of biodiversity. One can inspect the fence
distribution in the grid maps, but we have not yet analyzed total fence
perimeter, or area fenced, or other summary measure of fence prevalence. 

In the future, we would want to represent the mass ungulate die-off mentioned
in the Introduction. This tragic die-off occurred as a consequence of drought
following large expansions in fencing. We would thus want to measure herbivore
survivability following drought as a function of different socio-ecological 
contexts leading to different fence prevalences.


\subsection{Implementation}

Our agent-based model is implemented in the Julia programming 
language~\cite{Perkel2019} using the Agents.jl package~\cite{Datseris2022}. 
We used R~\cite{RLang} and the ggplot2~\cite{Wickham2016} package for
visualization and analysis. The simulation code is publicly available on
GitHub:~\url{https://github.com/mt-digital/fences}.


\begin{figure}[h]
  \caption{\textbf{Model trophic network (A),
      model biodiversity dynamics (B), and socio-environmental dynamics of 
    fence construction and drought (C).} In the biodiversity dynamics (B), a
    simulated drought at time $t=42$ (see grass die-off in center panel of C)
    destabilizes the ecosystem and herbivore populations crash without
    recovery, while livestock also crashes, but recovers.  Fences (pink polygon
    overlays) are constructed over time in response to carnivore predation of
    livestock and encroachment by other herbivores. Vegetation is shown in
    green, with darker cells indicating less vegetation due to consumption by
    livestock and other herbivores; or also due to drought, which occurs at
    $t=42$ in (C). Time $t$ is in months. 
}
  \centering
  \includegraphics[width=0.95\textwidth]{Figures/all_draft_figs.pdf}
  \label{fig:composition}
\end{figure}


\section{Analysis (sketch)}

We sketch here what our modeling framework enables through a prototype, toy
analysis of trophic and environmental dynamics following a drought. In our
model a drought is modeled by ``killing'' a fraction of the vegetation,
setting that fraction's ``grass'' content to 0.  

First, we initialized a 200x200 grid where vegetation in each cell is set to
its maximum of 1.0 (Figure~\ref{fig:composition}C, left panel). There are
twenty landholders in this model, five of which begin with fenced land. At
time $t=42$ a drought hits the landscape, represented by 75\% of all 
vegetation cells are set to 0 (Figure~\ref{fig:composition}C, center panel). 
By this time, $t=42$, almost all fences have
been constructed around the landholdings, with one more being constructed at 
$t=84$, where we can see vegetation has started to regrow following the
simulated drought. 

The drought does not have immediate effects on herbivore (such as ungulates) 
prevalence, but eventually does lead to a crash and decline of herbivore
populations around $t=100$. Livestock suffer a similar crash, but are
protected from predation and do not continue to decline following the crash
(Figure~\ref{fig:composition}B).


\section{Discussion}

We developed a spatial model of environmental and trophic dynamics
that depend on fencing decisions by landholders. The model can track
the interactions of thousands of animals, and the behavior of dozens of
landholders. 

A current challenge in using the model is the requirement and tuning of 
several parameters. There are still more parameters in the model code
than included in this draft for the sake of brevity. This appears to be
unavoidable. Still, some thought should be put in to consider what's
necessary in this model in light of a possible goal of this model to 
understand the increase in risk of massive ungulate die-off.  

Fence construction in the model is currently probabilistic. This is fine
for a zeroth approximation. In future iterations predation, encroachment,
legislation, or observing or hearing about other landholders nearby or 
more distant family
members~\cite{Pisor2024} construct fences could increase
the probability a landholder builds a fence.

This approach provides a framework for taming inherent complexity in
socio-economic systems by combining cultural evolution approaches to
socially-influenced decision making~\cite{PisorLansingMagargal2023},
which could provide a foundation for a more expansive application of cultural
evolutionary theory to the socio-ecological system of biodiversity
protection in the Greater Mara and similar 
ecosystems~\cite{Brooks2018,Currie2024}.


\printbibliography

\end{document}
