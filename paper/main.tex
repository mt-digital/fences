% \documentclass[12pt,reqno]{amsart}
\documentclass{article}
% \usepackage[letterpaper, margin=1.1in]{geometry}
\usepackage{amsfonts} 
\usepackage{amsmath}
\usepackage{amsthm}
\usepackage{amscd}
\usepackage{amsfonts}
\usepackage{amssymb}
\usepackage{mathrsfs}
\usepackage[bookmarks, colorlinks=true, allcolors=blue]{hyperref}
\usepackage{graphicx, wrapfig}
\usepackage[usenames]{color}
\usepackage[top=1.25in, bottom=1.25in, left=1.25in, right=1.25in]{geometry}
% \usepackage{natbib}
%\usepackage{showkeys}
\usepackage{caption}
\usepackage{subcaption}
\usepackage{longtable}
\usepackage{dsfont}
\usepackage{wrapfig}
\usepackage{authblk}

\usepackage[sorting=none, style=numeric-comp]{biblatex}

\addbibresource{/Users/mt/workspace/Writing/library.bib}
\addbibresource{this.bib}


\setcounter{MaxMatrixCols}{10}

 \definecolor{red}{rgb}{1.0,0.0,0.0}
 \def\red#1{{\textcolor{red}{#1}}}
 \definecolor{blu}{rgb}{0.0,0.0,1.0}
 \def\blu#1{{\textcolor{blu}{#1}}}
 \definecolor{gre}{rgb}{0.03,0.50,0.03}
 \def\gre#1{{\textcolor{gre}{#1}}}
 \definecolor{darkviolet}{rgb}{0.58, 0.0, 0.83}
\def\dvio#1{{\textcolor{darkviolet}{#1}}}

% \def\Re{\operatorname{Re}}
\sloppy
\DeclareMathOperator*{\argmin}{arg\,min}
\DeclareMathOperator*{\argmax}{arg\,max}
\newcommand{\ud}{{\, \mathrm{d}}}
\newcommand{\R}{\mathbb{R}}
\newcommand{\N}{\mathbb{N}}



\newtheoremstyle{mytheorem}% name
{6pt}%Space above
{6pt}%Space below
{\itshape}%Body font
{-0pt}%Indent amount 1
{\large \scshape}% Theorem head font
{}%Punctuation after theorem head
{1em}%Space after theorem head 2
{}%Theorem head spec (can be left empty, meaning "normal")

\newtheoremstyle{myremark}% name
{6pt}%Space above
{10pt}%Space below
{\rm}%Body font
{-0pt}%Indent amount 1
{\large \scshape}% Theorem head font
{}%Punctuation after theorem head
{1em}%Space after theorem head 2
{}%Theorem head spec (can be left empty, meaning "normal")




\theoremstyle{mytheorem}
\newtheorem{Theorem}{Theorem}
\newtheorem{Definition}[Theorem]{Definition}
\newtheorem{Proposition}[Theorem]{Proposition}
\newtheorem{Lemma}[Theorem]{Lemma}
\newtheorem{Corollary}[Theorem]{Corollary}
\newtheorem{Hypothesis}[Theorem]{Hypothesis}
\newtheorem{Corollaryoftheproof}[Theorem]{Corollary (of the proof)}


\theoremstyle{myremark}
\newtheorem{Remark}{Remark}
\newtheorem{Example}{Example}
\newtheorem{Notation}{Notation}





% \bibpunct{(}{)}{; }{a}{,}{,}
% \setcitestyle{authoryear}    % Imposta lo stile di citazione come autore-anno
\parskip=5pt

% \linespread{1.5}\selectfont

% \usepackage{float}
% \usepackage{caption}
% \captionsetup[subfigure]{position=bottom}
% \usepackage{subfig}
% \usepackage{subfloat}

\definecolor{mt-orange}{RGB}{240, 96, 0}
\newcommand{\mt}[1]{{\textcolor{mt-orange} {({\tiny MT:} #1)}}}
\definecolor{your-color-name}{RGB}{60,50,168}
\newcommand{\YourInitialsHere}[1]{{\textcolor{your_color_name} {({\tiny YI:} #1)}}}

\begin{document}

\title{Group-level homophily in metapopulation social networks can optimize the diffusion of innovations}

\author{Matthew A. Turner$^{*,1}$, SFI Biodiversity Team$^{2}$ \\ 
  {\small $^1$Stanford Doerr School of Sustainability,
Environmental Social Sciences, Stanford University, USA \\ \vspace{0.5em}
$^2$Earth} \\ \vspace{0.5em}
{\small $^*$Correspondence: \href{mailto:maturner@stanford.edu}{maturner@stanford.edu}}}
\maketitle

\begin{abstract}
  \noindent
  In the Greater Mara, Kenya, across the past few decades, conservationists and
  policymakers have struggled to develop policies and practices that are robust to
  predictable and unpredictable impediments, and effectively optimize precious
  political, legal, and enforcement resources for biodiversity conservation. Indeed,
  biodiversity conservation worldwide struggles to meet critical targets, including
  the United Nations' Sustainable Development Goals 14 and 15, to protect, restore,
  and promote sustainable use of marine (Goal 14) and terrestrial ecosystems (Goal
  15).  We suggest that this is because any decades-long policy-based course of
  action in socio-ecological systems for is subject to uncertainty, stochasticity,
  and path-dependence of outcomes characteristic of the socio-ecological systems in
  which the policies. There are predictable impediments to biodiversity: these
  include personal-level cognitive biases or incentives of voters and policymakers
  such as the temptation of lucrative poaching or illegal grazing.  Another
  predictable impediment is if social networks are not sufficiently structured,
  biodiversity-protecting behaviors may fail to percolate through relevant
  populations via interpersonal observation and social. These impediments are indeed
  \emph{predictable} in some sense, but only in contexts that meet some set of
  theoretical assumptions and conditions, with observations necessarily limited to a
  finite number of experimental laboratory or real-world settings. We clearly cannot
  perform real-world experiments on the Mara, or any other biodiversity preserve,
  because we do not get more than one, and it must be saved now. Here we develop a
  strategy for anticipating a range of possible outcomes for biodiversity
  interventions by simulating the complex local-, individual-, and relatively
  shorter time-scale dynamics of stakeholder decision-making and trophic networks
  that are affected by regional, state, or national policy. For instance, we
  demonstrate that a push to incorporate more rangeland into conservancies can tip
  unincorporated pastoralists to build fences around their landholdings, which may
  eventually limit wild herbivore mobility by encircling herbivores in the reserve,
  which can become deadly. In one tragic example, in the case of a sudden severe
  drought, many landholders had been fencing their land over the course of many
  years, which had not been a problem since water had been plentiful. When the
  drought hit, wildebeest needed to migrate for water, but were unable to due to
  fence construction that only became an obvious problem after that tragic event.
  We explain how agent-based modeling can represent local dynamics, while these
  dynamics are constrained by national- and regional-level policy. These policies 
  are aimed at optimizing biodiversity outcomes over longer time scales, 
  which could be calculated, for example obtained through dynamic
  optimization that accounts for actor and stakeholder incentives. More local-,
  shorter time-scale predictions can be used to update optimal policies, which
  in turn generate new smaller-scale simulation outcomes, and so on so that
  the models at each scale contribute to a strategy for implementing long-term
  \emph{and} adaptive policy to protect biodiversity.
  \mt{Just an idea, haven't done this; I would very much welcome
  ideas on a Kenyan-focused or -inspired idea for a proof of concept like this.}
\end{abstract}


\section{Introduction: predicting effects of environmental policy on 
  biodiversity (in the Greater Mara, Kenya)}

Biodiversity has both intrinsic and economic value, but is increasingly threatened
due to expanding human populations, overexploitation of land and game animals,
climate change and pollution, and invasive species encroachment~\cite{Bellard2022}.
Government policy, social influence, and personal preference all affect land use
decisions on different time, population, and geospatial scales 
(plant crops, raise cattle, etc.)~\cite{Eitzel2020,Løvschal2021}.
Fence construction is one important anthropocene feature that threatens
biodiversity~\cite{Packer2013} and whole 
ecosystems~\cite{Løvschal2017,Løvschal2022}. 
In turn, changes in biodiversity affect land use, resulting in a complex, coupled
dynamical system. In the Mara, 
ungulates are particularly threatened~\cite{Ogutu2009}, with fencing being primary
contributor to wildlife declines generally~\cite{Ogutu2016}. 
There are many reasons why people subdivide land and build
fences~\cite{Mwangi2007}, though these may change depending on local context.
Similarly, the risks posed by different fences are ostensibly
different in different contexts~\cite{Bellard2022}, perhaps even within
a single ecosystem like the Mara. 

Because of the need to understand land use and
fence construction on different time, population, and geospatial scales, we
outline here a self-consistent, two-pronged modeling strategy. One 
model is intended to understand, or predict alternative, outcomes over 
longer time scales, and larger population and geospatial scales,
based on a mean-field policy optimizaiton. This approach would be used to 
understand how government policies that lasted decades affected the distribution of
land use, the size of wildlife populations, and the total area fenced, for example.

Of course, these larger-scale dynamics occuring over decades emerges from
individual-level decisions and local interpersonal interactions, such as 
when a farmer in the Mara must decide whether to build a fence or not based on
the amount of wildlife concentration on their land, which is a function of whether or
not neighbors built fences, joined conservancies, etc.~\cite{}. Local taxes,
regulations, and other laws could have different effects in different places.
To account for such local details and predict short-term outcomes in geographic areas
among fewer people, we also develop a proof-of-concept agent-based model where
individual-level incentives and interpersonal influence shape fence building
behaviors, which in turn affect biodiversity, which in turn affects trophic network
dynamics. 





\printbibliography

\end{document}
